\paragraph{}{Pomysł polega na przerzedzaniu list $L_i$ przez wyrzucanie elementów, które niewiele się od siebie różnią.}
\paragraph{}{Niech $O<\delta <1$. Z list $L$ utworzymy krótszą listę $L'$ taką, że dla każdego $y$ usuniętego z $L$ w $L'$ pozostaje element $z$ taki, że $\frac{y}{1+\delta}\leq z\leq y$ }
\paragraph{Przykład:\\}{
$$L = \{ 15, 16, 17, 25, 27, 28, 35, 36, 37 \},  \delta = 0.1 $$
$$L' = \langle 15, 17, 25, 28, 35 \rangle $$}
\paragraph{}{Następująca funkcja Trim(L, $\delta$) implementuje ten pomysł. L to jest posortowana rosnąc lista L = $\langle y_1, ..., y_m \rangle$.}
\begin{lstlisting}[caption={Trim(L,$\delta$)}]
$m \leftarrow |L|$
$L' \leftarrow \langle y_1 \rangle$
$last \leftarrow y_1$
for $i \leftarrow 2$ to m
	if $y_i > last (1 + \delta)$
		dolacz $y_i$ na koniec L'
		$last \leftarrow y_i$
return L'
\end{lstlisting}

\paragraph{Działa w czasie $\Theta(m)$}{Poprawność jak poprzednio.}
\paragraph{Schemat aproksymacyjny:}
\begin{lstlisting}[caption={ApproxSubsetSum(S,t,$\varepsilon$)}]
$n \leftarrow |S|$
$L_0 \leftarrow \langle 0 \rangle$
for $i \leftarrow 1$ to n
	$L_i \leftarrow MergeLists(L_{i-1},L_{i-1}+x_i)$
	usun z $L_i$ wszystkie elementy $>$ t
	$L_i \leftarrow Trim(L_i,\frac{\varepsilon}{2n})$
return $z^*$ najwiekszy element na liscie $L_n$
\end{lstlisting}
\paragraph{Przykład:}{$S=\lbrace 205, 203, 400, 200\rbrace \tab n=4$ \\ $t=650$ \\ $\varepsilon = 0,4 \tab \delta=\frac{\varepsilon}{2n}=\frac{0,4}{8}=0,05$}

\begin{enumerate}[start = 0]
\item $L_0 = \langle 0 \rangle$
\item $L_1 = \langle 0,205 \rangle$\\
	  $L_1 = \langle 0,205 \rangle\tab$usuwanie\\
	  $L_1 = \langle 0,205 \rangle\tab$trimowanie
\item $L_2 = \langle 0,203,205,408 \rangle$
	  $L_2 = \langle 0,203,205,408 \rangle\tab$usuwanie\\
	  $L_2 = \langle 0,203,408 \rangle\tab$trimowanie
\item $L_3 = \langle 0,203,400,408,603,808 \rangle$
	  $L_3 = \langle 0,203,400,408,603 \rangle\tab$usuwanie\\
	  $L_3 = \langle 0,203,400,603 \rangle\tab$trimowanie
\item $L_4 = \langle 0,200,203,400,403,600,603,803 \rangle$
	  $L_4 = \langle 0,200,203,400,403,600,603 \rangle\tab$usuwanie\\
	  $L_4 = \langle 0,200,400,600 \rangle\tab$trimowanie
\end{enumerate}
\paragraph{Optymalne rozwiązanie:}{$z^*=600\tab$ $205+203+200 = 608\tab$ $\frac{608}{600}<1+\varepsilon = 1,4$}