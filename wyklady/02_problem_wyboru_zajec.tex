\subsection{Problem wyboru zajęć}
\begin{itemize}
\item $S = \{ Z_1,...,Z_n \}$ - zbór zajęć używających tego samego zasobu (np. sali zajęć lub procesora) 
\item $s_i$ - czas rozpoczęcia zajęcia $z_i$
\item $t_i$ - czas zakończenia zajęcia $z_i$
\end{itemize}
Zajęcia $z_i$ i $z_j$ są \textbf{zgodne} jeśli $\langle s_i,t_i)$ $\cup$ $\langle s_j,t_j) = \emptyset$
\paragraph{Problem:}{ Wybrać największej liczności zbiór $A \subset S$ parami zgodnych zajęć.}
\begin{center}
\begin{tabular}{ c | c | c | c | c | c | c }
  $i$ & $1$	& $2$ & $3$ & $4$ & $5$ & $6$ \\ \hline
  $s_i$ & 0	&  2  & 4 	& 7 & 1 & 7 \\ \hline
  $t_i$ & 4	&  6  & 7 	& 10 & 3 & 9 \\  
\end{tabular}
\end{center}
%\includegraphics[width=\textwidth]{UI_screens/Graph.png}

Algorytm zachłanny w każdym kroku wybierze zajęcie zgodne z każdym z już wybranych zajęć, które kończy się najwcześniej, żeby zostawić miejsce dla jak największej liczby kolejnych zajęć.

Sortujemy zajęcia niemalejąco względem czasów zakończenia (można to zrobić w czasie $O(n \log n))$. Od tej pory zakładamy, że $t_1 \leq t_2 \leq ... \leq t_n$

\begin{center}
\begin{tabular}{ c | c | c | c | c | c | c }
  $i$ & $1$	& $2$ & $3$ & $4$ & $5$ & $6$ \\ \hline
  $s_i$ & 1	&  0  &  2	&  4  &  7  & 7 \\ \hline
  $t_i$ & 3	&  4  &  6	&  7  &  9  & 10 \\  
\end{tabular}
\end{center}

\subsubsection{Algorytm}
gdzie $s$,$t$ - wektory
\begin{lstlisting}
$n \leftarrow length(S)$
$A \leftarrow \{ z_1 \}$
$j \leftarrow 1$				//indeks ostatnio dodanego zajecia
for $i \leftarrow 2$ to $n$
	if $s_i \geq t_j$ 
    	$A \leftarrow A \cup \{ z_i \}$
    	$j \leftarrow i$
return A
\end{lstlisting}

Algorytm działa poprawnie tzn. zwraca zbiór zajęć parami zgodnych, bo dołącza zajęcie $z_i$ do zbioru rozwiązań $A$ tylko wtedy, gdy $s_i \geq t_j$, gdzie $t_j$ jest czasem zakończenia zajęcia wykonanego poprzednio.

\paragraph{Twierdzenie:} {Algorytm zawsze znajduje optymalne rozwiązania.
\paragraph{Dowód:} {Indukcja po n}

\subparagraph{$n = 1$}
	OK
\subparagraph{$n > 1$}{Pokażemy na początku, że zawsze istnieje rozwiązanie optymalne zawierające zajęcie $z_1$:

Niech $A_1$ będzie dowolnym rozwiązaniem optymalnym i niech $\vert A_1\vert = l$ 

Jeśli $z_1 \in A_1$ to OK.

Jeżeli $z_1 \not\in A_1$:

Niech $z_k$ będzie zajęciem z $A_1$ o najmniejszym indeksie. $\forall{j} \neq k, z_j \in A_1$ $s_j \geq t_k$ oraz $t_k \geq t_1$. 
Zatem $A = (A_1 - \lbrace z_k\rbrace ) \cup \lbrace z_1 \rbrace $ też jest rozwiązaniem optymalnym. $A$ jest rozwiązaniem optymalnym zawierającym $z_1$. Zatem zawsze istnieje rozwiązanie optymalne zawierające $z_1$.}
\newline

Rozważmy nasz problem dla zbioru $S' = \lbrace z_i \in S: s_i \geq t_1 \rbrace$ Dla każdego $z_j \in A - \lbrace z_1 \rbrace$ $z_j \in S'$, bo $A$ jest rozwiązaniem dla $S$. Pokażemy, że $A - \lbrace z_1 \rbrace$ jest rozwiązaniem optymalnym problemu dla $S'$. Gdyby istniało rozwiązanie $B$ dla $S'$ liczniejsze od $A - \lbrace z_1 \rbrace$, to $\lbrace z_1 \rbrace \cup B$ byłoby rozwiązaniem problemu dla $S$ liczniejszym od $A$.
$$\vert S' \vert \leq \vert S \vert$$
Z założenia indukcyjnego algorytm znajduje rozwiązanie optymalne $B'$ dla $S'$ o liczności $\vert A \vert - 1 = l - 1$
Algorytm znajduje dla zbioru $S$ rozwiązanie $A = \lbrace z_1 \rbrace \cup B'$ o liczności $\vert \lbrace z_1 \rbrace \cup B' \vert = 1 + l - 1 = l$, więc $A$, jest rozwiązaniem optymalnym.

 
\textbf{Złożoność obliczeniowa}
\begin{lstlisting}
linia 1		$O(n)$
linia 2 	$O(1)$
linia 3 	$O(1)$
petla w linii 4 wykonuje sie $O(n)$ razy
	wykonanie tej petli zajmuje $O(1)$ czasu
\end{lstlisting}

$\Theta(n)$ jeśli zajęcia są posortowane względem czasu zakończenia \\
\underline{$O(n \log n )$ złożoność całego rozwiązania.}

