Dane: $S=\lbrace s_1,\ ...,\ s_n \rbrace \subset N,\ t\in N$

Poszukujemy zbioru $S'\subset S\ \Sigma_{x\in S'}x \rightarrow max\Sigma_{x\in S'}\leq t$

Jest to problem NP zupełny.

\paragraph{Dowód:}{Problem jest w NP. DLa instancji (S,t) certyfikatem jest $S' \subset S$ Możemy sprawdzić w czasie wielomianowym czy $\forall_{x \in S'} x = t$}

Pokażemy, że 3-SAT $\leq_p$ SUBSET\_SUM 

$I$ - formuła 3-SAT w postaci CNF
$$ x_1, x_2, x_3 - zmienne$$
$$ c_1, c_2, c_3 - klauzule$$
$$ c_i = (x_a \lor x_b \lor x_c)$$

Zakładamy że żadna klauzula nie zawiera jednocześnie zmiennej i jej negacji oraz żę każda zmienna występuje w przynajmniej jednej klauzuli.

Załóżmy instancję $S,t)$ problemu SUBSET\_SUM dla $I$

Dla każdej zmiennej $x_i$ mamy 2 liczby $r_i$ oraz $r_i'$ n+m cyfrowe.

Dla każdej zmiennej $c_i$ mamy 2 liczby $\delta_i$ oraz $\delta_i'$ n+m cyfrowe.

$t$ jest liczbą n+m cyfrową. 1...(n)...14...(m)...4

Pierwsze $n$ cyfr etykietujemy zmiennymi, a ostatnich $m$ cyfr etykietujemy klauzulami.
$r_i$ i $r_i'$ mają 1 na pozycji etykietowanej zmiennej $x_i$.
Jeśli $x_i$ występuje w $c_j$ to $r_i$ jest 1 na pozycji etykietowanej $c_j$.
Jeśli negacja $x_i$ występuje w $c_j$ to $r_i'$ jest 1 na pozycji etykietowanej $c_j$. na pozostałych pozycjach $r_i$ i $r_i'$ mają 0.

$s_j$ ma 1 na na pozycji etykietowanej $c_j$ i 0 na pozostałych pozycjach.
$s_j'$ ma 2 na na pozycji etykietowanej $c_j$ i 0 na pozostałych pozycjach.
$S = {r_1, r_1', r_2, r_2', ..., r_m,r_m'}$
$(S,t)$ instancja problemu.

Liczby w $S$ są różne. $\sum_{s_i}$ ma na każdej pozycji cyfrę $\leq 6$.
z tego wynika że nie ma przenoszeń.
Redukcja działa w czasie wielomianowym.
$S$ ma $2(n+m)$ liczb $n+m$ cyfrowych każda z tych liczb można utworzyć w czasie wielomianowym. natomiast $t$ tworzymy w czasie $(2(n+m)+1)\cdot(n+m)$

$I$ jest spełnione, gdy mamy wartościowanie $I$ przy którym $I$ prawdziwa.
jeśli $x_i=0$ to do $S'$ bierzemy $r_i'$
jeśli $x_i=1$ to do $S'$ bierzemy $r_i$
$c_j$ jest spełnione, gdy suma $r_i, r_i'$  $i \in [n]$ z $S'$ ma na pozycji $c_j$  1,2 lub 3. wiec do $S'$ dobiermay odpowiednio $s_j$ oraz $s_j'$ lub $s_j'$ lub $s_j$
 
$S' \subset S$ sumuje się do $t$

do $S'$ należy dokładnie jedna z liczba $r_i, r_i'$ dla każdego $i \in [n]$
jeżeli $r_i  \in S'$ to $x_i = 0$
jeżeli $r_i' \in S'$ to $x_i = 1$

\paragraph{Przykład: }{
$$c_1=(x_1 \lor -x_2 \lor -x_3), $$
$$c_2=(-x_1 \lor -x_2 \lor -x_3),$$ 
$$c_3=(-x_1 \lor -x_2 \lor x_3), $$
$$c_4=(x_1 \lor x_2 \lor x_3),$$ 
$$I = (c_1 \land c_2 \land c_3 \land c4)$$
}

TABELA

Rozwiązanie istnieje dla $I  x_1=1 , x_2=0, x_3=0$

\paragraph{Lemat 1:}
$$S={x_1, ..., x_n}$$
$$P_0 = {0}$$
$$P_i = P_{i-1} \cup (P_{i-1} + x_i) \text{ dla } i=1, \ldots, n$$
Zbiór $P_n$ zawiera wszystkie możliwe sumy podzbiorów zbioru $S$

\paragraph{Dowód:}{Pokażemy przez indukcję po i, że $P_i$ składa się ze wszystkich możliwych sum podzbiorów z $S$.

$$i = 0 \tab P_0=\{0\} \sum_{x_i \in \emptyset} = 0$$

Załóżmy, że  $P_{i-1}$ składa się ze wszystkich możliwych sum podzbiorów $\{x_1, ..., x_{i-1}\}$ 

Jeśli $x_i \not\in S'$ to $\sum_{x_j \in S'} x_j \in P_{i-1} \subset P_i$

Jeśli $x_i \in S'$ to $\sum_{x_j \in S'} x_j = \sum_{x_j \in S' - \{x_i\}} (x_j) + x_i \in P_{i-1} + x_i \subset P_i$
 
Zatem $P_i$ zwraca wszystkie sumy podzbiorów $\{ x_1, ..., x_i\}$ i nie zawiera innych liczb, bo $P_i = P_{i-1} \cup (P_{i-1} + x_i)$ 
jeśli $x \in P_{i-1}$ to jest sumą podzbioru $\{ x_1, ..., x_i\} $
jeśli $x \in P_{i-1}+x_i$ to $x-x_i$ jest sumą podzbioru $\{ x_1, ..., x_{i-i}\} $
dla $i=n$ otrzymujemy tezę.

\paragraph{}{Załóżmy, że mamy procedurę MergeLists(L, L'), gdzie L' są posortowanymi listami, która zwraca posortowanymi listami, która zwraca posortowaną listę elementów z L i L' i usuwa duplikaty, działającą w czasie $O(|L|+|L'|)$.}

\paragraph{Algorytm dokładny:}
\begin{lstlisting}[caption={ExactSubsetSum(S,t)}]
$n \leftarrow |S| \tab$ /$\ast S=\lbrace x_1, ..., x_n \rbrace \ast$ /
$L_0 \leftarrow \langle 0 \rangle$
for $i \leftarrow 1$ to n
	$L_i \leftarrow MergeLists(L_{i-1}, L{i-1}+x_i)$
	usun z $L_i$ wszystkie elementy $>$ t
return Najwiekszy element na liscie $L_N$
\end{lstlisting}

\paragraph{Poprawność algorytmu}{wynika z Lematu 1.}

\paragraph{}{Długość listy $L_i$ może być wykładnicza względem i, nawet $2^i$}
\paragraph{}{Jaki jest rozmiar danych? $$|I|=\Theta(log t+ \Sigma^n_{i=1}log x_i)$$ Jeśli t i $x_i$ są liczbami n cyfrowymi to logt, log$x_i=\Theta(n)$, więc $|I|>\Theta(n+n^2)=\Theta(n^2).$}
\paragraph{}{Ponieważ ln może mieć długość $2^n$ algorytm jest wykładniczy.}
\paragraph{}{Jeśli t jest ograniczone przez wielomian względem n = $|S|$, czyli $t=O(n^p)$ dla pewnego wielomianu p, to długość list $L_i$ są ograniczone przez $t=O(n^p)$, więc algorytm jest wielomianowy.}
\paragraph{}{Pokażemy w pełni wielomianowy schemat aproksymacyjny dla tego problemu.}
