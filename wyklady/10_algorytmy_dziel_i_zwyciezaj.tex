\section{Algorytmy Dziel i zwyciężaj}
\subsection{Definicje}
Ten typ algorytmu składa się z dwóch etapów:
\begin{enumerate}
	\item DZIEL - Dzielimy problem na mniejsze rozłączne podproblemy, 
	\item ZWYCIĘŻAJ - te podproblemy rozwiązujemy rekurencyjnie.
	\item POŁĄCZ - konstruowanie rozwiązania całego problemu z rozwiązań podproblemów. 
\end{enumerate}

Zwykle zakładamy że:
\begin{itemize}
	\item dzielimy na co najmniej dwa podproblemy.
	\item problemy na które dzielimy są rozłączne.
\end{itemize}

\subsection{Mnożenie $n$ cyfrowych liczb naturalnych}
$X,Y$ - $n$-cyfrowe liczby naturalne

Metoda pisemna obliczenia $X \cdot Y$ $\leftarrow$ trzeba wykonać $\Theta(n^2)$ mnożeń liczb jednocyfrowych.

Szukamy najlepszej metody podziału na podproblemy. Podzielmy $X$ i $Y$ na dwie połowy - $X_L, X_R$ oraz $Y_L, Y_R$.
$$X = x_1,x_2...x_n = x_1...x_{\frac{n}{2}},x_{\frac{n}{2}+1}, ...x_n = 10^\frac{n}{2} \cdot X_L+X_R$$
analogicznie
$$Y = 10^\frac{n}{2} \cdot Y_L+Y_R$$
zatem iloczyn $X \cdot Y$ wynosi
$$X \cdot Y = (10^\frac{n}{2} \cdot X_L + X_R) \cdot (10^\frac{n}{2} \cdot Y_L + Y_R) = 10^n \cdot X_L Y_L + 10^\frac{n}{2} \cdot (X_L Y_R + X_R Y_L) + X_R Y_R $$

Niech $T(n)$ oznacza złożoność algorytmu mnożącego w ten sposób $X$ i $Y$. Dodawanie liczb n-cyfrowych zajmuje $O(n)$ czasu. Mnożenie przez $10^n$ jest równoważne z dopisywaniem $n$ zer, więc zajmuje O(n) czasu. Zatem $T(n) = 4 \cdot T(\frac{n}{2}) + O(n)$, gdzie $T(\frac{n}{2})$ jest złożonością dla rozwiązywania pojedynczego podproblemu, a $O(n)$ oznacza złożoność łączenia rozwiązań podproblemów.

Można to jednak zrobić sprytniej i zauważyć, że:
$$X_L Y_R + X_R Y_L =  X_L Y_L + (X_L - X_R)(Y_R - Y_L) +X_R Y_R$$
co daje nam wzór:
$$X \cdot Y = 10^n \cdot X_L Y_L + 10^\frac{n}{2} \cdot (X_L Y_L+ (X_L - X_R)(Y_R - Y_L) + X_R Y_R) + X_R Y_R $$

Zatem trzeba policzyć tylko trzy iloczyny: $X_L Y_L$, $X_R Y_R$ i $(X_L - Y_R)(Y_R - X_L)$. Stąd
$$T(n) = 3 \cdot T(\frac{n}{2}) + O(n)$$