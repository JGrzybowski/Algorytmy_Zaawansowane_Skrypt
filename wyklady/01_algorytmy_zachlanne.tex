\section{Algorytmy Zachłanne}
\subsubsection{Notacja asymptotyczna}
\(f,g,{\rm I\!N} \rightarrow {\rm I\!R}_+\)
\begin{itemize}
	\item \( g(n) = O(f(n))\) jeśli \(\exists{c}>0\) \(\exists{n_0}\in {\rm I\!N} \) \(\forall{n} \geq n_0\) \(g(n) \leq c \cdot f(n)\)
    \item \( g(n) = \Omega(f(n))\) jeśli \(\exists{c}>0\) \(\exists{n_0}\in {\rm I\!N} \) \(\forall{n} \geq n_0\) \(c \cdot f(n) \leq g(n)\)
    \item \( g(n) = \Theta(f(n))\) jeśli \(\exists{c_1,c_2}>0\) \(\exists{n_0}\in {\rm I\!N} \) \(\forall{n} \geq n_0\) \(c_1 \cdot f(n)\leq g(n) \leq c_2 \cdot f(n)\)
\end{itemize}
W naszych rozważaniach będziemy zakładać że tablicę o długości $n$ można posortować w czasie $n \log n$.

\subsubsection{Algorytmy Zachłanne}
Idea algorytmu zachłannego - w każdym kroku algorytmu dokonuje lokalnie najlepszego wyboru, czyli wyboru, który w danym momencie wydaje się być optymalny.