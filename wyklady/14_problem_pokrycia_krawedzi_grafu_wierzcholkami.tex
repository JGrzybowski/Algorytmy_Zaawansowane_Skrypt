zbiór $W\subset V(G)$ nazywamy pokryciem wierzchołkowym grafu $G$ jeśli każda krawędź zbioru ma co najmniej jeden koniec w tym zbiorze ($W$).\\
\\
\textbf{Przykład: }\\
Problem znaleźć pokrycie wierzchołkowe $G$ o minimalnej ilości.\\
\\
\subsubsection{Algorytm ApproxVertexCover (AVC)}
\begin{lstlisting}[caption={AVC(G)}]
$C \leftarrow \emptyset \tab$  /$\ast$ C - znalezione pokrycie $\ast$ /
$F \leftarrow E(G) \tab$  /$\ast$ F - zbior jeszcze ni pokrytych krawedzi $\ast$ /
while $F \neq \emptyset$
	wybierz dowolna krawedz $uv \in F$
	$C \leftarrow C \cup \lbrace u, v\rbrace$
	usun z $F$ kazda krawedz majaca koniec w $u$ lub $v$
return $C$
\end{lstlisting}
\textbf{Poprawność: }\\
\tab Algorytm zwraca pokrycie, bo w każdym kroku algorytm dodaje wierzchołki pokrywające nie pokryte krawędzie.\\
\\
\textbf{Złożoność: }\\
\begin{lstlisting}
linia 2$O(m)$ czasu
petla 3-6 da sie zaimplementowac tak, zeby dzilala w czasie $O(n+m)$
\end{lstlisting}
\begin{center}
Zatem AVC jest algorytmem 2-aproksymacyjnym.
\end{center}
\textbf{Twierdzenie }AVC jest algorytmem 2-aproksymacyjnym\\
Dowód:\\
\tab $c^{\ast}$ - liczebność optymalnego pokrycia\\
\tab $c$ - liczebność rozwiązania znalezionego przez algorytm\\
\\
Niech B będzie zbiorem krawędzi wybranych przez AVC w linii 4\\
\\
B jest skojarzeniem w G. Aby pokryć krawędzie tego skojarzenia trzeba do pokrycia wybrać $\geq \vert B\vert$ wierzchołków, zatem $c^{\ast} \geq \vert B\vert \wedge c = 2\vert B\vert \Rightarrow c = 2\vert B\vert \leq 2c^{\ast} \Rightarrow \frac{c}{c^{\ast}} \leq 2$
