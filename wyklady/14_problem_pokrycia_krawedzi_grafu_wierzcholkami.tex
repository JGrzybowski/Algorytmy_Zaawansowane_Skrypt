\subsection{Problem pokrycia krawędzi grafu wierzchołkami}
Zbiór $W \subset V(G)$ nazywamy pokryciem wierzchołkowym grafu $G$, jeśli każda krawędź zbioru ma co najmniej jeden koniec w $W$.

\paragraph{Problem: }Znaleźć pokrycie wierzchołkowe $G$ o minimalnej wielkości. Decyzyjna wersja tego problemu jest NP-zupełna.

\subsubsection{Algorytm ApproxVertexCover (AVC)}
\begin{lstlisting}[caption={AVC(G)}]
$C \leftarrow \emptyset \tab \tab $  /$\ast$ C - znalezione pokrycie $\ast$ /
$F \leftarrow E(G) \tab$  /$\ast$ F - zbior jeszcze nie pokrytych krawedzi $\ast$ /
while $F \neq \emptyset$
	wybierz dowolna krawedz $uv \in F$
	$C \leftarrow C \cup \lbrace u, v\rbrace$
	usun z $F$ kazda krawedz majaca koniec w $u$ lub $v$
return $C$
\end{lstlisting}

\paragraph{Poprawność: } Algorytm zwraca pokrycie, bo w każdym kroku algorytm dodaje wierzchołki pokrywające jeszcze nie pokryte krawędzie.

\paragraph{Złożoność: } $|V(G)| =  n, |E(G)| = m$

Przepisanie krawędzi w linii 2. zajmuje $O(m)$ czasu.
Pętlę z linii 3-6 da się zaimplementować tak, aby działała w czasie $O(n+m)$.

\paragraph{Twierdzenie }AVC jest algorytmem 2-aproksymacyjnym.

\paragraph{Dowód:}
$c^{\ast}$ - liczebność optymalnego pokrycia
$c$ - liczebność rozwiązania znalezionego przez algorytm

Niech B będzie zbiorem krawędzi wybranych przez AVC w linii 4.

B jest skojarzeniem w G. Aby pokryć krawędzie tego skojarzenia trzeba do pokrycia wybrać $\geq \vert B\vert$ wierzchołków, zatem $$c^{\ast} \geq \vert B \vert, c = 2\vert B\vert \implies c = 2\vert B\vert \leq 2c^{\ast} \implies \frac{c}{c^{\ast}} \leq 2$$

Zatem AVC jest algorytmem 2-aproksymacyjnym.
