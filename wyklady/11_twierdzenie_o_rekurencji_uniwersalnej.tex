\subsection{Twierdzenie o rekurencji uniwersalnej}
% z przykładem stosowania rekurencji do mnożenia macierzy 
% na przykładzie algorytmu Strassena 
\paragraph{Twierdzenie:}
Niech $a \geq 1, b > 1$, będą stałymi, $f(n) = \Theta(n^k)$\\
Niech: $T(n) = a \cdot T(\frac{n}{b})+f(n)$, gdzie $\frac{n}{b}$ oznacza $\lfloor \frac{n}{b} \rfloor $ lub $\lceil \frac{n}{b} \rceil$. Wtedy:
$$
  T(n)=\begin{cases}
               \Theta(n^{\log_b a}), \text{ dla } a > b^k \Leftrightarrow \log_b a>k\\
               \Theta(n^{\log_b a} \log_b n), \text{ dla } a=b^k \Leftrightarrow \log_b a=k\\
               \Theta(n^k), \text{ dla } a<b^k \Leftrightarrow \log_b a<k
		\end{cases}
$$
$$a^{\log_b n}=b^{\log_b a^{\log_b n}}=b^{\log_b n \cdot \log_b a}=b^{\log_b a \cdot \log_b n}=b^{\log_b n^{\log_b a}}=n^{\log_b a}$$

\paragraph{Lemat: }Niech $a\geq 1,\ b>1$ będą stałymi i niech $f(n)$ będzie funkcją nieujemną zdefiniowaną dla potęg $b$.
Jeśli: 
\[
T(n)=\begin{cases}
	\Theta(1), $ dla $ n=1 \\
    a \cdot T(\frac{n}{b}) + f(n),$ dla $ n=b^i, $ gdzie $i \in \mathbf{N}_+
\end{cases}
\]\\
to\
$$T(n) = \Theta(n^{\log_b a}) + \sum_{j=0}^{\log_b n-1} a^j\ f(\frac{n}{b^j})$$

\textbf{Dowód lematu: }
$$T(n)= f(n) + a \cdot T(\frac{n}{b})= f(n) + a \cdot (f(\frac{n}{b}) + a  \cdot T(\frac{n}{b^2})) = f(n) + a \cdot f(\frac{n}{b}) + a^2 \cdot T(\frac{n}{b^2}) = $$
$$= f(n) + a \cdot f(\frac{n}{b}) + a^2 \cdot f(\frac{n}{b^2}) + a^3  \cdot T(\frac{n}{b^3}) = ... =$$
$$= f(n) + a \cdot f(\frac{n}{b}) + a^2 \cdot f(\frac{n}{b^2}) + ... + a^{(log_b n)-1} \cdot T(\frac{n}{b^{(\log_b n) - 1}}) + a^{\log_b n} \cdot T(1)$$

Ponieważ $$a^{\log_b n}=n^{\log_b a}$$ możemy powiedzieć, że $$a^{\log_b n} \cdot T(1) = \Theta(n^{\log_b a})$$
oraz $$T(n) = \Theta(n^{\log_b a})+\sum_{j=0}^{(\log_b n)-1} a^j\ f(\frac{n}{b^j}))$$

\textbf{Dowód twierdzenia} dla przypadku gdy $n$ jest potęgą $b$ ($n=b^x$, gdzie $x \in \mathbf{N}_+$). W pozostałych przypadkach dowód jest bardziej techniczny.

$$\Theta(\sum_{j=0}^{(\log_b n)-1} a^j \cdot f(\frac{n}{b^j}))=\Theta(\sum_{j=0}^{(\log_b n)-1} a^j\cdot (\frac{n}{b^j})^k)$$

\begin{enumerate}
\item Jeśli $a > b^k$, tzn. $\log_b a > k$ to
	$$ \sum_{j=0}^{(\log_b n)-1} a^j \cdot (\frac{n}{b^j}))^k 
	= \sum_{j=0}^{(\log_b n)-1} a^j \cdot \frac{n^k}{b^{jk}}
	= n^k \cdot \sum_{j=0}^{(\log_b n)-1} (\frac{a}{b^k})^j 
	=
	$$
	$$= n^k \cdot \frac{1-(\frac{a}{b^k})^{\log_b n}}{1-\frac{a}{b^k}} 
	= n^k \cdot \frac{\frac{a^{\log_b n}} {b^{k \cdot \log_b n}}-1}{\frac{a}{b^k}-1} 
	= n^k \cdot \frac{\frac{a^{\log_b n} - b^{k \cdot \log_b n}}{b^{k \cdot \log_b n}}}{\frac{a-b^k}{b^k}} 
	=
	$$
	$$
	= n^k \cdot \frac{a^{\log_b n} - b^{\log_b n^k}}{a - b^k} \cdot  \frac{b^k}{b^{\log_b n^k}}
	= n^k \cdot \frac{n^{\log_b a}-n^k}{a - b^k} \cdot \frac{b^k}{n^k} 
	=	
	$$
	$$	
	= \frac{b^k}{a-b^k}(n^{\log_b a}-n^k) 
	= \Theta(n^{\log_b a})$$
	\textbf{Z Lematu} $T(n) = \Theta(n^{\log_b a}) + \Theta(n^{\log_b a}) = \Theta(n^{\log_b a})$
\item Jeśli $a = b^k$, tzn. $\log_b a=k$
	$$	\sum_{j=0}^{(\log_b n)-1} a^j \cdot (\frac{n}{b^j}))^k 
	= n^k \cdot \sum_{j=0}^{(\log_b n)-1} (\frac{a}{b^k})^j 
	= n^k \cdot \sum_{j=0}^{(\log_b n)-1} 1 
	= n^k \cdot \log_b n
	$$
	Zatem 
	$$T(n) = \Theta(n^{\log_b a}) + \Theta(n^k \cdot \log_b n) 
	= \Theta(n^{\log_b n} + n^{\log_b a} \cdot \log_b n) 
	= \Theta(n^{\log_b a} \cdot \log_b n) $$
\item Jeśli $a<b^k$, tzn. $\log_b a<k$. Tak samo jak w punkcie 1.
	$$\sum_{j=0}^{(\log_b n)-1} a^j\ (\frac{n}{b^j})^k = n^k\ \frac{n^k-n^{log_ba}}{b^k-a} \cdot \frac{b^k}{n^k} = \frac{b^k}{b^k-a}\cdot (n^k - n^{log_ba})=\Theta(n^k)$$
	\textbf{Z Lematu} $T(n) = \Theta(n^{log_ba}) + \Theta(n^k) = O(n^k)$
\end{enumerate}

\subsection{Złożoność algorytmu mnożenia liczb n-cyfrowych}
\paragraph{Dla zwykłego mnożenia} złożoność wynosiła
	$$T(n) = 4T(\frac{n}{2}) + O(n)$$
Zatem $a,b$ oraz $k$ mają następujące wartości:
	$$a=4 \tab b=2 \tab k=1 \tab 4>2^1$$
Zatem złożoność ostatecznie ma wartość
	$$T(n) = \Theta(n^{log_24})=\Theta(n^2)$$
Czyli taką samą jak zwyczajne mnożenie pisemne.

\paragraph{Dla sprytnego algorytmu} złożoność wynosiła
	$$T(n) = 3T(\frac{n}{2})+O(n)$$
Zatem $a,b$ oraz $k$ mają następujące wartości:
	$$a=3 \tab b=2 \tab k=1 \tab 3>2^1$$
Zatem złożoność ostatecznie ma wartość
	$$T(n)= \Theta(n^{\log_2 3}) = O(n^{1,59})$$

\subsection{Mnożenie Macierzy}
Zwykły algorytm mnożenia macierzy $n \times n$ działa w czasie $O(n^3)$

A, B $=$ macierze $n \times n$\\
$C = AB$ 

Dzielimy $A,B$ i $C$ na cztery kwadratowe macierze $\frac{n}{2} \times \frac{n}{2}$ w następujący sposób:
\[
AB = 
\begin{bmatrix}
    A_{11}	& A_{12} \\
    A_{21}  & A_{22} 
\end{bmatrix}\ 
\begin{bmatrix}
    B_{11}  & B_{12} \\
    B_{21}  & B_{22} 
\end{bmatrix}
=
\begin{bmatrix}
    C_{11}  & C_{12} \\
    C_{21}  & C_{22} 
\end{bmatrix} = C
\]
gdzie 
$$C_{11}= A_{11}B_{11}+A_{12}B_{21}$$
$$C_{12}= A_{11}B_{12}+A_{21}B_{22}$$
$$C_{21}= A_{21}B_{11}+A_{22}B_{21}$$
$$C_{22}= A_{21}B_{12}+A_{22}B_{22}$$

Zamieniamy jedno mnożenie macierzy $n \times n$ na:
\begin{itemize}
	\item 8 mnożeń macierzy $\frac{n}{2} \times \frac{n}{2}$ kosztujących w sumie $8 \cdot T(\frac{n}{2})$ czasu 
	\item 4 dodawania macierzy  $\frac{n}{2} \times \frac{n}{2}$ kosztujące w sumie $4 \cdot O((\frac{n}{2})^2) = O(n^2)$ czasu.
\end{itemize} 
Złożoność takiego sposobu liczenia możemy policzyć z twierdzenia o  rekurencji uniwersalnej
$$T(n)=8 \cdot T( \frac{n}{2} )+ O(n^2)$$
$a,b$ oraz $k$ mają wartości:
$$a=8 \tab b=2 \tab k=2 \tab 8>2^2$$
Zatem końcowa wartość złożoności to:
$$T(n)=\Theta(n^{log_28}) = \Theta(n^3)$$

\subsection{Algorytm Strassena}
Strassen zauważył, że można policzyć:
$$M_1=(A_{12}-A_{22})(B_{21}+B_{22})$$
$$M_2=(A_{11}-A_{22})(B_{11}+B_{22})$$
$$M_3=(A_{11}-A_{21})(B_{11}+B_{12})$$
$$M_4=(A_{11}-A_{12})B_{22}$$
$$M_5=A_{11}(B_{12}-B_{22})$$
$$M_6=A_{22}(B_{21}-B_{11})$$
$$M_7=(A_{21}+A_{22})B_{11}$$
wtedy: 
$$C_{11}=M_1+M_2-M_4+M_6$$
$$C_{12}=M_4+M_5$$
$$C_{21}=M_6+M_7$$
$$C_{22}=M_2-M_3+M_5-M_7$$

Stąd rekurencja:
$$T(n)=7 \cdot T(\frac{n}{2}) + O(n^2)$$
$a,b$ oraz $k$ mają wartości:
$$a=7 \tab b=2 \tab k=2 \tab 7>2^2$$
Zatem złożoność algorytmu Strassena to:
$$T(n)= \Theta(n^{log_27})=O(n^{2,81})$$