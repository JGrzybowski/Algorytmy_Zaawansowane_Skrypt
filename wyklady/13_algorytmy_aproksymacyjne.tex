\section{Algorytmy Aproksymacyjne}
\textbf{Algorytm aproksymacyjny (przybliżony)} - daje "prawie optymalne" rozwiązanie.

Algorytmy aproksymacyjne stosuje się tam, gdzie jest mała szansa na wielomianowy algorytm dokładny.

$\Pi$ - problem aproksymacyjny polegający na maksymalizacji lub minimalizacji pewnej funkcji celu.

Niech $c^{\ast} > 0$ będzie optymalnym rozwiązaniem problemu $\Pi$, a $c > 0$ wartością znalezioną przez algorytm aproksymacyjny $A$ dla $\Pi$.

Mówimy, że algorytm aproksymacyjny $A$ ma ograniczenie względne (współczynnik aproksymacji) $\rho (n)$ jeśli dla dowolnych danych o rozmiarze $n$ $$max(\frac{c}{c^{\ast}},\frac{c^{\ast}}{c}) \leq \rho (n)$$

$A$ nazywamy algorytmem $\rho(n)$ - aproksymacyjnym jeżeli 
\begin{center}
	$\begin{drcases}
		\text{dla problemu maksymalizacji } 0<c\leq c^{\ast}\\
		\text{dla problemu minimalizacji } 0<c^{\ast}\leq c
	\end{drcases} 
	\Rightarrow max(\frac{c}{c^{\ast}},\frac{c^{\ast}}{c})\geq 1 \Rightarrow \rho(n) \geq 1$
\end{center}

Błędem względnym rozwiązania $c$ nazywamy $$\frac{\vert c-c^{\ast} \vert}{c^{\ast}}$$

Mówimy, że algorytm aproksymacyjny $A$ posiada ograniczenia błędu  względnego $\varepsilon (n)$ jeśli dla danych o rozmiarze $n$ $$\frac{\vert c-c^{\ast}\vert}{c^{\ast}} \leq \varepsilon (n)$$ 
dla problemu maksymalizacji 
$$	\frac{\vert c^{\ast} - c\vert}{c^{\ast}} 
	= \frac{c^\ast - c}{c^\ast}
	\leq \frac{c^\ast - c}{c}
	= \frac{c^{\ast}}{c} - 1 
	\leq \rho(n) - 1 
	\tab 0 < c \leq c^{\ast} $$
dla problemu minimalizacji 
$$	\frac{\vert c - c^{\ast}\vert}{c^{\ast}} 
	= \frac{c - c^{\ast}}{c^\ast} 
	= \frac{c}{c^{\ast}} - 1 
	\leq \rho(n) - 1 
	\tab 0 < c^{\ast} \leq c $$

$$
\rho = sup_{n \in \mathbf{N}}
	\lbrace \rho(n) \rbrace 
\tab 
\varepsilon = sup_{n \in \mathbf{N}} 
	\lbrace\varepsilon (n) \rbrace
$$

\textbf{Mówimy wtedy, że współczynnik aproksymacji wynosi $\rho$ i że $A$ jest algorytmem $\rho$-aproksymacyjnym.}