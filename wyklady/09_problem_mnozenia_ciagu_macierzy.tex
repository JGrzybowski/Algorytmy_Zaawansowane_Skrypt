\subsection{Problem mnożenia ciągu macierzy}
\subsubsection{Przykład}
Dane są trzy macierze:
\begin{itemize}
\item $ A_1 (5 \times 10) $
\item $ A_2 (10 \times 3) $
\item $ A_3 (3 \times 2) $
\end{itemize}

Ile trzeba mnożeń skalarnych, aby obliczyć $A_1 \cdot A_2 \cdot A_3 = A_1 \cdot (A_2 \cdot A_3) = (A_1 \cdot A_2) \cdot A_3$?

Dla kolejności $A_1 \cdot (A_2 \cdot A_3)$ potrzeba $10 \cdot 3 \cdot 2 + 5 \cdot  10 \cdot 2 = 60 + 100 = 160$

Dla kolejności $(A_1 \cdot A_2) \cdot A_3$ potrzeba $5 \cdot 10 \cdot 3 + 5 \cdot 3 \cdot 2 = 150 + 30 = 180$

Pierwsza kolejność wymaga mniej obliczeń.

\subsubsection{Problem}
Dla danego ciągu macierzy $(A_1, ...,  A_n)$ gdzie $A_i$ ma wymiary $p_{i-1} \times p_i$ znaleźć kolejność wykonywania mnożeń tak, aby ilość mnożeń skalarnych była jak najmniejsza. 
Możliwych kolejności mnożeń jest tyle, ile możliwości rozmieszczenia nawiasów.

\textbf{Przykład} $A_1 \cdot A_2 \cdot A_3 \cdot A_4$
\begin{enumerate}
\item ($A_1 \cdot A_2) \cdot (A_3 \cdot A_4)$
\item (($A_1 \cdot A_2) \cdot A_3) \cdot A_4$
\item ($A_1 \cdot (A_2 \cdot A_3)) \cdot A_4$
\item $A_1 \cdot ((A_2 \cdot A_3) \cdot A_4)$
\item $A_1 \cdot (A_2 \cdot (A_3 \cdot A_4$))
\end{enumerate}

Niech $P(n)$ będzie liczbą możliwych ustawień nawiasów w mnożeniu $n$ macierzy. Niech ostatnie mnożenie ma postać $(A_1 \cdot ... \cdot A_k) \cdot (A_{k+1} \cdot ... \cdot A_n)$. Zatem:
$$
P(n) = 	\begin{cases}
		P(n) = \sum_{k=1}^{n-1} P(k) \cdot P(n-k) \text{ dla } n \geq 2 \\
		P(1) = 1
		\end{cases}
$$
To znane równanie rekurencyjne mające rozwiązanie $P(n) = C(n-1)$, gdzie $C(n) = \frac{1}{n+1}\binom{2n}{n}$ - liczba Cathalana. $C(n) = \Omega(\frac{4^n}{n^\frac{3}{2}})$.

Zauważmy, że jeżeli optymalne rozwiązanie ma postać $(A_1 \cdot ... \cdot A_k) \cdot (A_{k+1} \cdot ... \cdot A_n)$ to iloczyny $A_1 \cdot ... \cdot A_k$ oraz $A_{k+1} \cdot ... \cdot A_n$  też muszą być optymalne. W przeciwnym razie lepsze rozmieszczenia nawiasów w $A_1 \cdot ... \cdot A_k$ lub $A_{k+1} \cdot ... \cdot A_n$ dałoby lepsze rozmieszczenie nawiasów w $(A_1 \cdot ... \cdot A_k) \cdot (A_{k+1} \cdot ... \cdot A_n)$. Optymalne rozwiązanie zawiera więc optymalne rozwiązania problemów tego samego typu. Jest o jeden z warunków skuteczności programowania dynamicznego.


\textbf{Podproblemy} - rozmieszczenie nawiasów w iloczynie $A_i \cdot ... \cdot A_j$ gdzie $1 \leq i \leq j \leq n$. 

$m(i,j)$ - minimalna liczba mnożeń potrzebna do policzenia iloczynu $A_i \cdot ... \cdot A_j$.

Chcemy policzyć $m(1,n)$. Zakładamy, że $m(i,i) = 0$. 

Niech optymalne rozmieszczenie nawiasów ma postać $(A_i \cdot ... \cdot A_k) \cdot (A_{k+1} \cdot ... \cdot A_j)$ dla $1 \leq  i \leq k < j \leq n$. Otrzymujemy rekurencję: 
$$m(i,j) = min_{i \leq k < j} \{ m(i,k) +m(k+1,j) + p_{i-1} \cdot p_k \cdot p_j \}$$
Przez $s(i,j)$ oznaczamy tę wartość $k$ dla której osiągane jest to minimum. Innymi słowy, $s(i,j)$ jest miejscem ostatniego mnożenia w optymalnym rozwiązaniu podproblemu. 

Gdyby zaprogramować tę rekurencję, otrzymalibyśmy algorytm o złożoności wykładniczej. Ale mamy tylko $\binom{n}{2} + n = \Theta(n^2)$ podproblemów, czyli $\Theta(n^2)$ liczb $m(i,j)$ i każdą z nich liczymy tylko raz.
\begin{lstlisting}[caption={Algorytm MatrixChainOrder($p$)- algorytm wyznaczania kolejności mnożenia macierzy ($A_i$ ma wymiary $p_{i-1} \times p_i$}]
n <- length(p) - 1
for i <- 1 to n
	m(i,i) <- 0
for l <- 2 to n // l - liczba macierzy w podproblemie
	for i <- 1 to n+1-l
		j <- i+l-1
		m(i,j) <- infinity
		for k <- i to j-1
			q = m(i,k) + m(k+1,j) + p_{i-1} \cdot p_k \cdot p_j
			if(q < m(i,j)
				m(i,j) <- q
				s(i,j) <- k
return (m,s)

\end{lstlisting}
Złożoność wynosi $O(n^3)$

Konstruowanie najlepszego rozwiązania
\begin{lstlisting}[caption={MatrixChainMultiply($A,s,i,j$) - algorytm mnożenia macierzy}]
if j = 1 return A_i
X <- MatrixChainMultiply(A, ) //HELP
Y <- MatrixChainMultiply(   )//HELP
return MatrixMultiply(X,Y)
\end{lstlisting}

\subsection{Przykład}
$$ A_1 4 \times 5 $$
$$ A_2 5 \times 2 $$
$$ A_3 2 \times 1 $$
$$ A_4 1 \times 2 $$
$$ A_5 2 \times 3 $$

\begin{lstlisting}
MatrixChainMultiply(A,s,i,j)
if j = i return A_i
X <- MatrixChainMultiply(A,s,i,s(i,j))
Y <- MatrixChainMultiply(A,s,s(i,j)+1, j)
return MatrixMultiply(X,Y)
\end{lstlisting}

$$s(1,5) = 3$$
$$(A_1 * A_2 * A_3) * (A_4 * A_5)$$
$$s(1,3) = 1$$
$$s(4,5) = 4$$
$$(A_1 * (A_2 * A_3)) * (A_4 * A_5)$$
$$s(2,3) = 2$$
$$(A_1 * (A_2 * A_3)) * (A_4 * A_5)$$

Podstawy metody programowania dynamicznego
Aby algorytm programowania dynamicznego mógł dawać oczekiwane rezultaty muszą być spełnione dwa warunki:

\begin{enumerate}
\item Własność optymalnej podstruktury - optymalne rozwiązanie problemu można skonstruować z optymalnych rozwiązań podproblemów. Jest to cecha m.in. problemów rozwiązywalnych algorytmami rekurencyjnymi czy dziel i zwyciężaj.
\item Własność wspólnych podproblemów - sytuacja gdy, algorytm rekurencyjny wielokrotnie oblicza rozwiązanie tych samych podproblemów.
\end{enumerate}
Liczba podproblemów nie powinna być zbyt wielka, aby nie zaburzać złożoności algorytmu.

Spamiętywanie
W klasycznej metodzie programowania kdynamicznego ozwiązanie budujemy od dołu do góry tzn od najmniejszych podproblemóW do największych. Istneje inne podejście gdzie budujemy w rozwiązanie od góry do dołu, tak jak w rekurencji z tą różnicą, żę unikamy wielokreotnego liczenia wartości funkcji poprzez zapisywanie jej do pamięci za przy pierwszym liczeniu  i odczytywanie jej w następnych wywołaniach. TO podejście nazywamy spamiętywaniem.

Przykład w mnożeniu macirzy
Jak poprzednio umieszczamy rozwiązania podproblemów w tablicy m(i.j) Dopóki nie mamy rozwiązanego podproblemu (i,j)  m(i,j) = infinity. Po obliczeniu wartości m(i,j) umieszczamy ją w tablicy i używamy jej przy następnych wywołaniach bez obliczania.

\begin{lstlisting}
Algorytm LookupChain($p,i,j$) //$p$ - ciag wymiarow macierzy
if $m(i,j) < \infty$ return $m(i,j)$
if $j = i$ 
	$m(i,j)$ <- 0 
else
	for $k <- i$ to $j-1$
		q <- LookupChain($p,i,k$) + LookupChain($p,k+1,j$) + $p_{i-1}$ * $p_k$ * $p_j$
		if $q < m(i,j)$
			$m(i,j)$ <- $q$
return $m(i,j)$
\end{lstlisting}

\begin{lstlisting}
Algorytm MemorizedMatrixChain($p$)
$n$ <- length($p$) - 1
for $i$ <- 1 to $n$
	for $j$ <- $i$ to $n$
		$m(i,j)$ <- $\infty$
return LookupChain($p,i,n$)
\end{lstlisting}

Złożoność 
\begin{lstlisting}
$O(n)$
$O(n^2)$
$O(n^2)$
$O(n^2)$
\end{lstlisting}

Każde z $O(n^2)$ pól tablicy $m$ jest wypełniane tylko raz modyfikowane, potem odwołanie do (i,j) zajmuje stały czas.
Kiedy m(i,j) jest modyfikowane to zajmuje to $O(n)$ aby znaleźć jej najmniejszą wartość.
Całkowity czas działania algorytmu wynosi $O(n^3)$

