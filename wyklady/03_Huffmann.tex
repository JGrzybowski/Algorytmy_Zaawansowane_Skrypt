Zwykle znaki są reprezentowane za pomocą 8 bitów (np. ASCII). Ponieważ jednak częstości wystąpień znaków różnią się więc opłaca się reprezentować częściej występujące znaki za pomocą krótszych ciągów bitów, a rzadziej  występujące za pomocą dłuższych ciągów bitów.\\
\\
Dążymy do użycia jak najmniejszej ilości bitów.\\
\\
\textbf{Przykład:} W pliku występuje 58 znaków a, e, i, s, t, r, n\\
\begin{center}
\begin{tabular}{ c | c | c | c | c | c | c | c }
  $i$ & $a$	& $e$ & $i$ & $s$ & $t$ & $r$ & $n$ \\ \hline
  częstość wystąpień & 10 &  15  &  12	&  3  &  4  &  13 &  1 \\ \hline
  kod 1 (stała długość słowa kodowego) & 000 & 001 & 010 & 011 & 100 & 101 & 110 \\ \hline
  kod 2 (zmienna długość słowa kodowego) & 001 & 01 & 10 & 00000 & 0001 & 11 & 00001 \\ \hline
  kod 3 (zmienna długość słowa kodowego) & 001 & 01 & 10 & 00010 & 0001 & 11 & 00001 \\  
\end{tabular}
\end{center}

W przypadku stałej długości słów kodowych zakodowanych plik ma: $58*3=174$ bity.\\
W przypadku zmiennej długości słów kodowych zakodowany plik ma: $3*10+2*15+2*12+5*3+4*4+2*13+5*1 = 146$ bitów.\\
Oszczędzamy około 15%.\\
\\
Przy kodowaniu ze zmienną długością słów kodowych chcemy, żeby spełniany był warunek, że żadne słowo nie jest prefiksem (początkiem) innego słowa. Takie kody nazywamy prefiksowymi.\\
\\
\textbf{Kod 3 0001001 można zdekodować na dwa sposoby:}
\begin{itemize}
	\item 00010 01 se
	\item 0001 001 ta 
\end{itemize}
tutaj 0001 jest prefiksem słowa 00010 czyli kod t jest prefiksem kodu s.\\
\\
\textbf{Reprezentacja kodów za pomocą drzew}\\
\textbf{Kod 1}\\
%\includegraphics[width=\textwidth]{UI_screens/Graph.png}
\textbf{Kod 3}\\
%\includegraphics[width=\textwidth]{UI_screens/Graph.png}
\textbf{Kod 2}\\
%\includegraphics[width=\textwidth]{UI_screens/Graph.png} 
\underline{stw.} Żadne słowo nie jest prefiksem innego słowa $\Leftrightarrow$ w drzewie reprezentującym ten kod wszystkie znaki są liśćmi.