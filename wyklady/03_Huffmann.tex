Zwykle znaki są reprezentowane za pomocą 8 bitów (np. ASCII). Ponieważ jednak częstości wystąpień znaków różnią się więc opłaca się reprezentować częściej występujące znaki za pomocą krótszych ciągów bitów, a rzadziej  występujące za pomocą dłuższych ciągów bitów.\\
\\
Dążymy do użycia jak najmniejszej ilości bitów.\\
\\
\textbf{Przykład:} W pliku występuje 58 znaków a, e, i, s, t, r, n\\
\begin{center}
\begin{tabular}{ l | c | c | c | c | c | c | c }
  	 										& $a$	& $e$ 	& $i$ 	& $s$ 	& $t$ 	& $r$ 	& $n$ 	\\ \hline
  częstość wystąpień 						& 10	&  15  	& 12	&  3  	&  4	&  13	&  1 	\\ \hline
  kod 1 (stała długość słowa kodowego) 		& 000 	& 001 	& 010	& 011 	& 100	& 101	& 110 	\\ \hline
  kod 2 (zmienna długość słowa kodowego) 	& 001 	& 01 	& 10	& 00000 & 0001	& 11	& 00001 \\ \hline
  kod 3 (zmienna długość słowa kodowego) 	& 001 	& 01 	& 10	& 00010 & 0001	& 11	& 00001 \\  
\end{tabular}
\end{center}

W przypadku stałej długości słów kodowych zakodowanych plik ma: $58*3=174$ bity.\\
W przypadku zmiennej długości słów kodowych zakodowany plik ma: $3*10+2*15+2*12+5*3+4*4+2*13+5*1 = 146$ bitów.\\
Oszczędzamy około 15%.\\
\\
Przy kodowaniu ze zmienną długością słów kodowych chcemy, żeby spełniany był warunek, że żadne słowo nie jest prefiksem (początkiem) innego słowa. Takie kody nazywamy prefiksowymi.\\
\\
\textbf{Kod 3 0001001 można zdekodować na dwa sposoby:}
\begin{itemize}
	\item 00010 01 se
	\item 0001 001 ta 
\end{itemize}
tutaj 0001 jest prefiksem słowa 00010 czyli kod t jest prefiksem kodu s.\\
\\
\textbf{Reprezentacja kodów za pomocą drzew}\\
\textbf{Kod 1}\\
%\includegraphics[width=\textwidth]{UI_screens/Graph.png}
\textbf{Kod 3}\\
%\includegraphics[width=\textwidth]{UI_screens/Graph.png}
\textbf{Kod 2}\\
%\includegraphics[width=\textwidth]{UI_screens/Graph.png} 
\underline{stw.} Żadne słowo nie jest prefiksem innego słowa $\Leftrightarrow$ w drzewie reprezentującym ten kod wszystkie znaki są liśćmi.

\begin{tikzpicture}[->,>=stealth',level/.style={sibling distance = 5cm/#1,
  level distance = 1.5cm}] 
\node [treenode] {58}
    child{ node [treenode] {33} 
        child{ node [treenode] {18} 
        	child{ node [treenode] {8} 
				child{ node [treenode] {4} 
					child{ node [treenode] (s) {s} }
					child{ node [treenode] (n) {n} } 				
				}
				child{ node [treenode] (t) {t} }            	
            }
			child{ node [treenode] (a) {a} }	
		}  
		child{ node [treenode] (e) {e} }
    }
    child{ node [treenode] {25}
            child{ node [treenode] (i) {i} }
            child{ node [treenode] (r) {r} }
	}
; 
\end{tikzpicture}

Etykietujemy wierzchołki wewnętrzne drzewa w taki sposób, aby każdy wierzchołek $x$ otrzymywał etykietę będącą sumą ilości wystąpień znaków będących liśćmi w poddrzewie o korzeniu $x$. Niech $C$ będzie alfabetem i  $f(x)$ liczbą występowania znaku $c$, $d_T$ głębokością liścia $c$ w drzewie $T$ (równą długości słowa kodowego reprezentującego $c$). Liczba bitów potrzebnych do zakodowania całego pliku wynosi $C(T) = \sum_{c \in C} f(c) \cdot d_t(c)$. 

Kod prefiksowy zdefiniowany przez drzewo $T$ jest optymalny jeśli liczba $B(T)$ ???
Drzewo ukorzenione  jest binarne jeśli każdy wierzchołek ma co najmniej 2 dzieci.
Drzewo binarne jest regularnej jeśli każdy wewnętrzny wierzchołek ma dokładnie 2 dzieci.

\textbf{Twierdzenie} Drzewo binarne reprezentujące kod optymalny jest regularne.
\textbf{Dowód} $T$ - Drzewo reprezentując optymalny kod prefiksowy. Przy puśćmy, że $T$ nie jest regularne. Wtedy istnieje w T wewnętrzny wierzchołek z dokładnie jednym dzieckiem.
$w$ - pojedyncze dziecko
$v$ - rodzic $w$, nie będący korzeniem drzewa.
$u$ - rodzic $v$
%Ilustracja
Tworzymy nowe drzewo $T'$ poprzez usunięcie $v$ z drzewa i dodanie krawędzi $uw$ o ile $v$ nie jest korzeniem.

Jeżeli $B(T) < B(T')$ zachodzi sprzeczność, ponieważ $T$ miało być drzewem optymalnym.

\subsection{Algorytm tworzenia drzewa reprezentującego optymalny kod prefiksowy}
Algorytm tworzenia drzewa reprezentującego optymalny kod prefiksowy dla $(C,f)$. 
Zaczynamy od $|C|$ liści, następnie wykonujemy $|C|-1$ operacji scalania zaczynając od najrzadziej występujących znaków (będą one leżały najgłębiej w drzewie). Będziemy używać kolejki priorytetowej z klasami $f(c)$. Kolejkę priorytetową można zaimplementować jako kopiec binarny. Wtedy poszczególne operacje będą miały następujący koszt:
\begin{itemize}
\item insert $O(\log n)$ 
\item extract\_min $O(\log n)$
\item create $O(n)$
\end{itemize}
gdzie $n$ będzie liczbą elementów w kolejce.

\subsubsection{Przykład}
\begin{enumerate}
\item n:1 s:3 t:4 a:10 i:12 r:13 e:15
%Ilustracja
\end{enumerate}

\subsection{Algorytm Huffmana}
Załóżmy istnienie funkcji $left(z), right(z)$ - zwracające dzieci wierzchołka $z$.
\begin{lstlisting}[caption={Huffman(C,F)}]
n <- |C|
Q <- create_queue(C)
for i <- i to n-1
	z <- create_new_node
	x <- left(z) <- extract_min(Q)
	y <- right(x) <- extract_nim(Q)
	f(z) <- f(x) + f(y)
	insert(Q,z)
\end{lstlisting}
Złożoność algorytmu Huffmana:
\begin{lstlisting}
$O(1)$
$O(1)$
$n-1$ razy
	wnetrze petli $\log n$ razy
\end{lstlisting}

Algorytm konstruuje drzewa binarne

\textbf{Lemat 1.} Niech $x$ i $y$ będą parą znaków z $C$ o najmniejszej liczbie wystąpień. Istnieje optymalny kod prefiksowy, w którym kody dla $x$ i $y$ mają tę samą długość i różnią się tylko wartością ostatniego bitu.

\textbf{Dowód} Znajdujemy drzewo reprezentujące kod optymalny, w którym liście reprezentujące $x$ i $y$ są bliźniakami i są najgłębiej w drzewie. Niech $T$ będzie drzewem reprezentującym kod optymalny i niech $a$ i $b$ będą liśćmi bliźniakami najgłębiej w T. 
Bez straty ogólności możemy założyć:
$$ f(x) \leq f(y); f(a) \leq f(b) $$
ponieważ $f(x)$ i $f(y)$ są najmniejszymi liczbami:
$$ f(x) \leq f(a); f(y) \leq f(b) $$
Niech $T'$ będzie drzewem powstałym z $T$ przez zamianę miejscami $x$ z $a$
$$ B(T) - B(T') = \sum_{c \in C} f(c) \cdot d_T(c) - \sum_{c \in C} f(c) \cdot d_{T'}(c) $$
$$ = f(x) \cdot d_T(x) + f(a) \cdot d_T(a) - f(x) \cdot d_{T'}(x) - f(a) \cdot d_{T'}(a) $$
$$ = f(a) \cdot (d_T(a) - d_T(x)) + f(x) \cdot (d_T(x) - d_T(a)) $$ 
$$ = (f(a) + f(x)) \cdot (d_T(a) - d_T(x)) \geq 0$$
Zatem $B(T') \leq B(T)$ a jednocześnie $T$ jest optymalne. Zatem $B(T') = B(T)$ więc $T'$ też jest drzewem reprezentującym kod optymalny. Analogicznie zamieniamy $y$ z $b$.

\textbf{Z lematu 1.} wynika, że możemy zacząć kodowanie optymalnego kodu prefiksowego od dwóch znaków najrzadziej występujących. Zachłanność wyboru polega na wybieraniu obiektów o najmniejszej częstości.

\textbf{Lemat 2.} $A$ - alfabet. Niech $x$ i $y$ będą parą znaków z $A$ o  najmniejszej częstości. Niech $A'= (A - \{x,y\}) \cup \{z\}$ gdzie $z \not\in A$. Funkcję $f$ definiujemy jak dla $A$ a dodatkowo $f(z) = f(x) + f(y)$.
$T'$ - drzewo reprezentujące kod dla $A'$
$T$ - drzewo powstałe przez zastąpienie liścia $z$ wierzchołkiem wewnętrznym z liśćmi $x$ i $y$.
Drzewo $T$ jest drzewem reprezentującym kod optymalny dla A.

\textbf{Dowód} Dla $c \in A - \{x,y\}$ zachodzi $d_T(c) = d_{T'}(c)$ oraz $d_T(x) = d_T(y) = d_{T'}(z)+1$, stąd 
$$B(T) = \sum_{c \in A}(f(c) \cdot d_T(c)) = \sum_{c \in A-\{x,y\}}(f(c) \cdot d_T(c)) + f(x) \cdot d_T(x) + f(y) \cdot d_T(y)$$
$$ = \sum_{c \in A-\{x,y\}}(f(c) \cdot d_T(c)) + (d_{T'}(z) +1) \cdot (f(x) + f(y)) = \sum_{c \in A-\{x,y\}}(f(c) \cdot d_T(c)) + (d_{T'}(z) +1) \cdot (f(x))$$
$$ = \sum_{c \in A'}(f(c) \cdot d_{T'}(c)) + f(x) + f(y) = B(T') + f(x) + f(y) \implies B(T') = B(T) - f(x) - f(y)$$

Przypuśćmy, że $T$ nie reprezentuje optymalnego kodu dla A. Istnieje $T"$ reprezentujące kod optymalny dla A i $B(T"  < B(T)$. Na mocy \textbf{lematu 1.} możemy wybrać $T"$ tak, aby $x$ i $y$ były w $T'$ bliźniakami położonymi w drzewie. Niech $T'''$ będzie drzewem powstałym z $T"$ przez wyrzucenie $x$ i $y$ i dodanie $z$ oraz $f(z) = f(x) + f(y)$. $T'''$ jest drzewem reprezentującym pewien kod prefiksowy dla $A'$. Analogicznie możemy pokazać, że $B(T")$ ???

\subsection{Twierdzenie o poprawności Algorytmu Huffmana}
\textbf{Twierdzenie} Algorytm Huffmana zawsze znajduje optymalny kod prefiksowy.
\textbf{Dowód} Indukcja po $n = |C|$
Dla $n = 1,2$ twierdzenie jest prawdziwe.
Dla $n > 2$:
Po pierwszym scaleniu otrzymujemy alfabet $C' = (C - {x,y}) \cup \{z\}$, funkcję $f$ taką, że $f(z) = f(z) + f(y)$, a dla pozostałych wartości argumentu jest zdefiniowana jak przed scaleniem. $|C'| = |C| -1$
Z założenia indukcyjnego algorytmu znajdującego drzewo $T'$ reprezentujące kod optymalny dla $C'$ algorytm znajduje dla $C$ drzewo $T$ powstałe z $T'$ przez zastąpienie liście z wierzchołkiem wewnętrznym $z$ dziećmi $X$i $y$ z \textbf{lematu 2.} T reprezentuje optymalny kod prefiksowy dla $C$




