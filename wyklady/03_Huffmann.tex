\subsection{Kody Huffmann'a}
Zwykle znaki są reprezentowane za pomocą 8 bitów (np. ASCII). Ponieważ jednak częstości wystąpień znaków różnią się, opłaca się reprezentować częściej występujące znaki za pomocą krótszych ciągów bitów, a rzadziej  występujące za pomocą dłuższych.

Dążymy do użycia jak najmniejszej ilości bitów.

\textbf{Przykład:} W pliku występuje 58 znaków a, e, i, s, t, r, n
\begin{center}
\begin{tabular}{ l | c | c | c | c | c | c | c }
  	 										& $a$	& $e$ 	& $i$ 	& $s$ 	& $t$ 	& $r$ 	& $n$ 	\\ \hline
  częstość wystąpień 						& 10	&  15  	& 12	&  3  	&  4	&  13	&  1 	\\ \hline
  kod 1 (stała długość słowa kodowego) 		& 000 	& 001 	& 010	& 011 	& 100	& 101	& 110 	\\ \hline
  kod 2 (zmienna długość słowa kodowego) 	& 001 	& 01 	& 10	& 00000 & 0001	& 11	& 00001 \\ \hline
  kod 3 (zmienna długość słowa kodowego) 	& 001 	& 01 	& 10	& 00010 & 0001	& 11	& 00001  
\end{tabular}
\end{center}

Przy stałej długości słów kodowych zakodowanych plik ma: $58*3=174$ bity. 

Przy zmiennej długości słów kodowych zakodowany plik ma: $3 \cdot 10+2 \cdot 15+2 \cdot 12+5 \cdot 3+4 \cdot 4+2 \cdot 13+5 \cdot 1 = 146$ bitów. Oszczędzamy około 16$\%$.

Przy kodowaniu ze zmienną długością słów kodowych chcemy, żeby spełniony był warunek, że żadne słowo nie jest prefiksem (początkiem) innego słowa. Takie kody nazywamy prefiksowymi. 

\textbf{Słowo 0001001 można zdekodować za pomocą kodu 3. na dwa sposoby:}
\begin{itemize}
	\item 00010 01 se
	\item 0001 001 ta 
\end{itemize}
Tutaj 0001 jest prefiksem słowa 00010 czyli kod t jest prefiksem kodu s. \textbf{Zatem kod 3. nie jest kodem prefiksowym.}

\textbf{Reprezentacja kodów za pomocą drzew}\\
\textbf{Kod 1}
%\includegraphics[width=\textwidth]{UI_screens/Graph.png}

\textbf{Kod 3}
%\includegraphics[width=\textwidth]{UI_screens/Graph.png}

\textbf{Kod 2}
%\includegraphics[width=\textwidth]{UI_screens/Graph.png} 

\paragraph{Stwierdzenie} Żadne słowo nie jest prefiksem innego słowa wtedy i tylko wtedy, gdy w drzewie reprezentującym ten kod wszystkie znaki są liśćmi.
\begin{center}
\begin{tikzpicture}[->,>=stealth',level/.style={sibling distance = 5cm/#1,
  level distance = 1.5cm}] 
\node [treenode] {58}
    child{ node [treenode] {33} 
        child{ node [treenode] {18} 
        	child{ node [treenode] {8} 
				child{ node [treenode] {4} 
					child{ node [treenode] (s) {s} }
					child{ node [treenode] (n) {n} } 				
				}
				child{ node [treenode] (t) {t} }            	
            }
			child{ node [treenode] (a) {a} }	
		}  
		child{ node [treenode] (e) {e} }
    }
    child{ node [treenode] {25}
            child{ node [treenode] (i) {i} }
            child{ node [treenode] (r) {r} }
	}
; 
\end{tikzpicture}
\end{center}

Etykietujemy wierzchołki wewnętrzne drzewa w taki sposób, aby każdy wierzchołek $x$ otrzymywał etykietę będącą sumą ilości wystąpień znaków będących liśćmi w poddrzewie o korzeniu $x$. 

Niech:
\begin{itemize}
\item $C$ będzie alfabetem
\item $f(c)$ będzie liczbą wystąpień znaku $c$
\item $d_T(c)$ będzie głębokością liścia $c$ w drzewie $T$ (równą długości słowa kodowego reprezentującego $c$)
\end{itemize}  Liczba bitów potrzebnych do zakodowania całego pliku wynosi 
$$B(T) = \sum_{c \in C} f(c) \cdot d_T(c)$$

Kod prefiksowy zdefiniowany przez drzewo $T$ jest optymalny jeśli liczba $B(T)$ jest najmniejsza możliwa.

Drzewo ukorzenione jest \textbf{binarne} jeśli każdy wierzchołek, nie będący liściem, ma co najwyżej 2 dzieci.

Drzewo binarne jest \textbf{regularne} jeśli każdy wewnętrzny wierzchołek ma dokładnie 2 dzieci.

\paragraph{Twierdzenie}{Drzewo binarne reprezentujące kod optymalny jest regularne.}

\paragraph{Dowód\\}{}
Załóżmy, że:
\begin{itemize}
\item $T$ - Drzewo reprezentujące optymalny kod prefiksowy. Przypuśćmy, że $T$ nie jest regularne. Wtedy istnieje w T wewnętrzny (nie będący korzeniem ani liściem) wierzchołek z dokładnie jednym dzieckiem.
\item $w$ - pojedyncze dziecko
\item $v$ - rodzic $w$, nie będący korzeniem drzewa.
\item $u$ - rodzic $v$
\end{itemize}
%Ilustracja

Tworzymy nowe drzewo $T'$ poprzez usunięcie $v$ z drzewa i dodanie krawędzi $uw$ o ile $v$ nie jest korzeniem. $T'$ jest optymalnym kodem prefiksowym. 
$B(T) < B(T')$ ponieważ usunęliśmy jeden wierzchołek w drzewie, zachodzi sprzeczność, ponieważ $T$ miało być drzewem optymalnym.

\subsubsection{Algorytm tworzenia drzewa reprezentującego optymalny kod prefiksowy}
Algorytm tworzenia drzewa reprezentującego optymalny kod prefiksowy dla $(C,f)$. 

Zaczynamy od $|C|$ liści, następnie wykonujemy $|C|-1$ operacji scalania zaczynając od najrzadziej występujących znaków (będą one leżały najgłębiej w drzewie). Będziemy używać kolejki priorytetowej z klasami $f(c)$. 

Kolejkę priorytetową można zaimplementować jako kopiec binarny. Wtedy poszczególne operacje będą miały następujący koszt:
\begin{itemize}
\item insert $O(\log n)$ 
\item extract\_min $O(\log n)$
\item create $O(n)$
\end{itemize}
gdzie $n$ będzie liczbą elementów w kolejce.

\paragraph{Przykład}
\begin{enumerate}
\item n:1 s:3 t:4 a:10 i:12 r:13 e:15
%Ilustracja
\end{enumerate}

\subsubsection{Algorytm Huffmana}
Załóżmy istnienie funkcji $left(z), right(z)$ - zwracające dzieci wierzchołka $z$.
\begin{lstlisting}[caption={Huffman(C,F)}]
n <- |C|
Q <- create_queue(C)
for i <- 1 to n-1
	z <- create_new_node
	x <- left(z) <- extract_min(Q)
	y <- right(x) <- extract_min(Q)
	f(z) <- f(x) + f(y)
	insert(Q,z)
\end{lstlisting}

Algorytm konstruuje drzewa binarne.

\textbf{Złożoność algorytmu Huffmana:}
\begin{lstlisting}
$O(1)$
$O(1)$
$n-1$ razy
	wnetrze petli $\log n$
\end{lstlisting}
\underline{Ostateczna złożoność algorytmu to $O(n \log n)$}

\subsubsection{Twierdzenie o poprawności Algorytmu Huffmana}
\textbf{Lemat 1.} Niech $x$ i $y$ będą parą znaków z $C$ o najmniejszej liczbie wystąpień. Istnieje optymalny kod prefiksowy, w którym kody dla $x$ i $y$ mają tę samą długość i różnią się tylko wartością ostatniego bitu.

\textbf{Dowód} Znajdujemy drzewo reprezentujące kod optymalny, w którym liście reprezentujące $x$ i $y$ są bliźniakami i są najgłębiej w drzewie. Niech $T$ będzie drzewem reprezentującym kod optymalny i niech $a$ i $b$ będą liśćmi bliźniakami najgłębiej w T. 
Bez straty ogólności możemy założyć:
$$ f(x) \leq f(y) \land f(a) \leq f(b) $$
ponieważ $f(x)$ i $f(y)$ są najmniejszymi wartościami funkcji $f$:
$$ f(x) \leq f(a) \land f(y) \leq f(b) $$
Niech $T'$ będzie drzewem powstałym z $T$ przez zamianę miejscami $x$ z $a$
$$ B(T) - B(T') = \sum_{c \in C} f(c) \cdot d_T(c) - \sum_{c \in C} f(c) \cdot d_{T'}(c) $$
$$ = f(x) \cdot d_T(x) + f(a) \cdot d_T(a) - f(x) \cdot d_{T'}(x) - f(a) \cdot d_{T'}(a) $$
$$ = f(a) \cdot (d_T(a) - d_T(x)) + f(x) \cdot (d_T(x) - d_T(a)) $$ 
$$ = (f(a) + f(x)) \cdot (d_T(a) - d_T(x)) \geq 0$$
Zatem $B(T') \leq B(T)$ a jednocześnie $T$ jest optymalne. Zatem $B(T') = B(T)$ więc $T'$ też jest drzewem reprezentującym kod optymalny. Analogicznie zamieniamy $y$ z $b$.

\paragraph{Z lematu 1.} {wynika, że możemy zacząć kodowanie optymalnego kodu prefiksowego od dwóch znaków najrzadziej występujących. Zachłanność wyboru polega na wybieraniu obiektów o najmniejszej częstości.}

\paragraph{Lemat 2.} {$A$ - alfabet. Niech $x$ i $y$ będą parą znaków z $A$ o  najmniejszej częstości. 
\\ Niech $A'= (A - \{x,y\}) \cup \{z\}$ gdzie $z \not\in A$. 
\\ Funkcję $f$ definiujemy jak dla $A$ a dodatkowo $f(z) = f(x) + f(y)$.

$T'$ - drzewo reprezentujące kod optymalny dla $A'$

$T$ - drzewo powstałe z $T'$ przez zastąpienie liścia $z$ wierzchołkiem wewnętrznym z dziećmi $x$ i $y$.

Drzewo $T$ jest drzewem reprezentującym kod optymalny dla A.}

\paragraph{Dowód} Dla $c \in A - \{x,y\}$ zachodzi $d_T(c) = d_{T'}(c)$ oraz $d_T(x) = d_T(y) = d_{T'}(z)+1$, stąd 

$$B(T) = \sum_{c \in A}(f(c) \cdot d_T(c)) = \sum_{c \in A-\{x,y\}}f(c) \cdot d_T(c) + f(x) \cdot d_T(x) + f(y) \cdot d_T(y)$$

$$ = \sum_{c \in A-\{x,y\}}f(c) \cdot d_T(c) + (d_{T'}(z) +1) \cdot (f(x) + f(y)) = \sum_{c \in A-\{x,y\}}f(c) \cdot d_T(c) + f(z) \cdot d_{T'}(z) + f(x)+ f(y)$$

$$ = \sum_{c \in A'}f(c) \cdot d_{T'}(c) + f(x) + f(y) = B(T') + f(x) + f(y) \implies B(T') = B(T) - f(x) - f(y)$$

Przypuśćmy, że $T$ nie reprezentuje optymalnego kodu dla A. 

Istnieje $T''$ reprezentujące kod optymalny dla A i $B(T'') < B(T)$. Na mocy \textbf{lematu 1.} możemy wybrać $T''$ tak, aby $x$ i $y$ były w $T''$ bliźniakami najgłębiej położonymi w drzewie. 

Niech $T'''$ będzie drzewem powstałym z $T''$ przez wyrzucenie $x$ i $y$ i dodanie $z$ oraz $f(z) = f(x) + f(y)$. $T'''$ jest drzewem reprezentującym pewien kod prefiksowy dla $A'$. 

Analogicznie możemy pokazać, że $B(T''') = B(T'') - f(x) - f(y)$. Wtedy $B(T''') = B(T'') - f(x) - f(y) < B(T) - f(x) - f(y) = B(T')$. 

Otrzymujemy sprzeczność $B(T''') < B(T')$ ponieważ $T'$ miało reprezentować kod optymalny dla $A'$.

\paragraph{Twierdzenie} Algorytm Huffmana zawsze znajduje optymalny kod prefiksowy.
\paragraph{Dowód} {Indukcja po $n = |C|$}

Dla $n = 1,2$ twierdzenie jest prawdziwe.

Dla $n > 2$:

Po pierwszym scaleniu otrzymujemy alfabet $C' = (C - {x,y}) \cup \{z\}$, funkcję $f$ taką, że $f(z) = f(z) + f(y)$, a dla pozostałych argumentów jest zdefiniowana jak przed scaleniem. $x$ i $y$ są najrzadziej występującymi znakami w C. 
$$|C'| = |C| - 1$$
Z założenia indukcyjnego algorytmu znajdującego drzewo $T'$ reprezentujące kod optymalny dla $C'$. Algorytm znajduje dla $C$ drzewo $T$ powstałe z $T'$ przez zastąpienie liścia $z$ wierzchołkiem wewnętrznym z dziećmi $x$i $y$ z \textbf{lematu 2.} $T$ reprezentuje optymalny kod prefiksowy dla $C$




