\subsubsection{Definicja}
$S = {z_1, ... ,z_n}$ - zbiór zadań każde o $t_{i_j}=1$ \\
$d_1, ... , d_n$,  $d_i \in \{1,...,n\}$ - deadline'y zadań \\
$w_1, ... , w_n$, $w_i \in \{1,...,n\}$ - kary za nie wykonanie zadania $z_i$ do chwili $t=d_i$ \\
Jeżeli zadanie zostanie ukończone po deadlinie, nazywamy je \textit{spóźnionym}, w przeciwnym przypadku zadanie jest nazywane \textit{terminowym}. \\

W problemie szeregowania zadań szukamy permutacji z najmniejszą karą. \\

\textbf{Lemat 1.} 
Każde uszeregowanie można doprowadzić do uszeregowania o tej samej sumie kar, w którym:
\begin{itemize}
	\item zadania terminowa poprzedzają opóźnione,
	\item zadania terminowe występują w kolejności niemalejącej względem deadlinów.
\end{itemize}  

\textbf{Dowód}
\begin{enumerate}
	\item{Niech: 
	\begin{itemize}
		\item $z_i$ będzie ukończone w chwili $l_i$ i spóźnione ($l_i > d_i$),
		\item $z_j$ będzie ukończone w chwili $l_j$ i terminowe 
		\item $z_i$ jest wykonywane przed $z_j$ ($l_i < l_j$)
	\end{itemize}
	Po zamianie $z_j$ z $z_i$:
	\begin{itemize}
		\item $z_i$ jest ukończone w chwili $l_j > l_i > d_i$, nadal opóźnione.
		\item $z_j$ jest ukończone w chwili $l_i \leq l_j \leq d_j$, nadal terminowe.
	\end{itemize}
	}
	
	\item{Niech $z_i, z_j$ będą zadaniami terminowymi ukończonymi odpowiednio w chwilach $l_i \leq d_i , l_j \leq d_j $. $l_i < l_j$ oraz $d_j < d_i$.
	Po zamianie $z_j$ z $z_i$:
	\begin{itemize}
		\item $z_i$ jest ukończone w chwili $l_j \leq l_i \leq d_i$, nadal terminowe.
		\item $z_j$ jest ukończone w chwili $l_i \leq l_j \leq d_j$, nadal terminowe.
	\end{itemize}
	}
\end{enumerate}

Definiujemy $N_t(A) = |\{z_i \in A : d_i \leq t \} | $ -  liczba zadań których $d$ nie jest większe od $t$. Zbiór zadań A nazywamy niezależnym, jeśli daje się go uszeregować tak, że wszystkie zadania są terminowe. 
\\

\textbf{Lemat 2.} 
Dla każdego $A \subset S$ następujące warunki są równoważne:
\begin{enumerate}
	\item $A$ jest niezależne
	\item $N_t(A) \leq t$ dla $t = 1,...,n$
	\item jeśli zadania w $A$ są uszeregowane względem deadlinów, to żadne zadanie nie jest opóźnione.
\end{enumerate}  

\textbf{Dowód}
\begin{itemize}
	\item{ $1 \implies 2$: 
	Jeśli $N_t(A) > t$ dla pewnego $t$ to do chwili $t$ trzeba by ukończyć więcej niż t zadań. Stąd nie da się szeregować zadań z $A$ tak, aby wszystkie były terminowe, więc $A$ nie jest niezależne.}
	\item{ $2 \implies 3$: 
	Szeregujemy zadania w $A$ malejąco względem deadlinów. Gdyby pewne zadanie $z_t$ w tym uszeregowaniu było spóźnione i kończone w chwili $;_t$ to byłoby co najmniej $l_t$ zadań o deadlinie $\leq d_t$. $z_t$ jest spóźnione, więc $l_t > d_t$.}
	\item{ $3 \implies 1$: 
	Szukamy takiego podzbioru niezależnego $A$ tak, aby suma kar dla elementów spoza $A$ była jak najmniejsza. $\sum_{z_i \not\in A} w_i = \sum_{z_i \in S} w_i - \sum_{z_i \in A} w_i$ co oznacza maksymalizację sumy $\sum_{z_i \in A} w_i$}
\end{itemize}

\textbf{Twierdzenie} 

$S$ - zbiór zadań i jednostkowych czasach wykonania 

$A$ - rodzina zbiorów niezależnych 

$(S,A)$ jest matroidem 


\textbf{Dowód}
\begin{enumerate}
	\item $S$ jest niepustym skończonym zbiorem
	\item $A \subset S$
	\item $A\in\mathbb{A}$, $B \subset A$ \\
	Uszeregowanie B powstałe z właściwego uszeregowanie przez wykreślenie z uszeregowania $A$ elementów spoza $B$.
	\item $A,B \in \mathbb{A}, |B| > |A|$ \\
	Za pomocą $k, 0\leq k \leq n$ oznaczamy największe $t$ takie, że $N_t(B) \leq N_t(A)$. $k$ jest dobrze zdefiniowane, ponieważ $N_0(B) = 0 = N_0(A)$. $N_n(B) = |B| >  |A| = N_n(A)$, zatem  $0\leq k \leq n$. \\
$\begin{drcases}
	N_{k+1}(B) > N_{k+1}(A) \\
	N_k(B) = N_k(A)
\end{drcases}$ istnieje zadanie $x \in B - A$ o deadlinie równym $k+1$
Niech $A' = A \cup \{x\}$. Pokażemy, że $A'$ jest niezależny używając \textbf{Lematu 2}. 
\begin{itemize}
	\item Ponieważ $A$ jest niezależne, dla $1 \leq t \leq k$ mamy $N_t(A') = N_t(A) \leq t$
	\item Ponieważ $B$ jest niezależne, dla $k+1 \leq t \leq k$ mamy $N_t(A') = N_t(A)+1 \leq N_t(B) \leq t$
\end{itemize}
\end{enumerate}
Zatem $(S,A)$ jest matroidem. Z Tw. Rado Edmondsa wiemy, że algorytm Greedy zwraca optymalne rozwiązanie. Dla problemu szeregowania zadań linijkę 4. w algorytmie możemy przerobić, tak aby działała w czasie $O(n)$. Zatem złożoność całego algorytmu wynosi $O(n^2)$.

\begin{center}
\begin{tabular}{ c | c | c | c | c | c | c | c | c } 
  	 	& $z_1$	& $z_2$ & $z_3$ & $z_4$	& $z_5$	& $z_6$ & $z_7$ & $z_8$	\\ \hline
  $d_i$	& 3		& 2		& 2		& 3		& 1		& 4		& 6		& 5		\\ \hline
  $w_i$	& 80	& 70	& 60	& 50	& 40	& 30	& 20	& 10	\\ 
\end{tabular}
\end{center}
\begin{enumerate}
	\item $\{ z_1 \}$
	\item $\{ z_2, z_1, \}$
	\item $\{ z_3, z_2, z_1, \}$
	\item $\{ z_3, z_2, z_1, \}$
	\item $\{ z_3, z_2, z_1, \}$
	\item $\{ z_3, z_2, z_1, z_6, \}$
	\item $\{ z_3, z_2, z_1, z_6, z_7 \}$
	\item $\{ z_3, z_2, z_1, z_6, z_8, z_7 \}$
\end{enumerate}

