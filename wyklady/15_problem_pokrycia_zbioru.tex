\subsection{Problem pokrycia zbioru}
$X$ - Zbiór skończony

$\mathcal{F} \subset 2^{X}$ - $\mathcal{F}$ rodzina podzbioru $X$ i tak, że $$\bigcup_{s\in \mathcal{F}}s = X$$ 
Czyli każdy element z $X$ należy do co najmniej jednego zbioru z $\mathcal{F}$.

\textbf{Definicja: }Rodzinę $\mathcal{A}\subset \mathcal{F}$ nazywamy pokryciem zbioru $X$ jeśli $$\bigcup_{s\in \mathcal{A}}s = X$$
\\
\textbf{Instancja: }$(X:\ \mathcal{F})$ gdzie $\mathcal{F} \subset 2^{X}$ takie, że $$\bigcup_{s\in \mathcal{F}}s = X$$

\textbf{Problem: }Należy znaleźć minimalnej liczebności pokrycie zbioru $X$, zbiorami z rodziny $\mathcal{F}$

\subsubsection{Algorytm GreedySetCover (GSC)}
\begin{lstlisting}[caption={GreedySetCover(G)}]
$U \leftarrow X \tab$  /$\ast$ $U$ - zbior jeszcze nie pokrytych elementow zbioru X $\ast$/
$\mathcal{A} \leftarrow \emptyset \tab$  /$\ast$  $\mathcal{A}$ - znalezione pokrycie $\ast$ /
while $U \neq \emptyset$
	wybierz $S \in \mathcal{F}$ maksymalizujacy $\vert S \cap U\vert$
	$\mathcal{A} \leftarrow \mathcal{A} \cup \lbrace S\rbrace$
	$U \leftarrow U - S$
return $\mathcal{A}$
\end{lstlisting}

\paragraph{Poprawność: }
Algorytm zwraca poprawne pokrycie, ponieważ na początku każdego przejścia pętli $U$ zawiera elementy jeszcze nie pokryte i w każdym przejściu pętli zostaje pokryty co najmniej jeden niepokryty element.

\textbf{Złożoność: }
\begin{lstlisting}
Liczba przejsc petli 3-6 jest $\leq \vert X\vert$ i $\leq \vert \mathcal{F} \vert$ wiec jest $\leq min\lbrace \vert X \vert,\ \vert \mathcal{F}\vert \rbrace$
Wnetrze petli mozna zaimplementowac zeby dzialalo w czasie $O(\vert X\vert \vert \mathcal{F}\vert)$
\end{lstlisting}
\begin{center}
Złożoność czasowa GSC wynosi $O(\vert X\vert \ast \vert\mathcal{F}\vert \ast min\lbrace \vert X \vert,\ \vert \mathcal{F}\vert \rbrace)$
\end{center}
\textbf{Twierdzenie: }GreedySetCover jest wielomianowym algorytmem $\rho(n)$ - aproksymacyjnym, gdzie $\rho(n) = H(max\lbrace\vert S\vert :S\in \mathcal{F}\rbrace)$, $H(m) = \lbrace 1 + \frac{1}{2} + ... +\frac{1}{m}\rbrace$, $H(0)=0$\\
\\
\textbf{Dowód: }$\mathcal{A}^{\ast}$ - optymalne pokrycie znalezione przez algorytm.\\
\tab Niech $S_i$ będzie zbiorem dodanym do $\mathcal{A}$ w w i-tym przebiegu pętli\\\\
Przekazujemy koszt $c_x=\frac{1}{\vert S_i-(S_1\cup ... \cup S_{i-1})\vert}$ każdemu elementowi $X\in S_i = (S_1 \cup ... \cup S_{i-1})$ (pokrytemu po raz pierwszy przez $S_i$)\\
\\
\tab W każdym przejściu pętli wszystkie z $S_i - (S_1 \cup ... \cup S_{i-1})$ dostają koszt $\frac{1}{\vert S_i-(S_1\cup ... \cup S_{i-1})\vert}$ więc całkowity koszt przyznawany w i-tym przejściu pętli wynosi 1.\\\\
\textbf{WYC} $\sum_{x\in X}c_x=\vert \mathcal{A}\vert$\\
\\
Pokażemy, że $\sum_{x\in S}c_x \leq H(\vert S\vert)$ dla każdego $S\in \mathcal{F}$\\
\\
Dla każdego $S\in \mathcal{F}$ oraz $i=1,\ 2,\ ...,\ \vert\mathcal{A}\vert$ definiujemy $n_i = \vert S - (S_1 \cup ... \cup S_i)\vert$ (liczba elementów w S nie pokrytych po i-1 przejściu pętli)\\
\tab $n_0=\vert S\vert$\\
\\
Niech k będzie najmniejszym indeksem takim, że $n_k=0$\\
\tab To znaczy, że S jest pokryty przez $S_1 \cup S_2 \cup ... \cup S_k$\\
Mamy $n_0\geq n_1\geq n_2\geq ...\geq n_{\vert\mathcal{A}\vert}$\\
\tab $n_{i-1}$ - $n_1$ jest liczbą elementów z S pokrytych po raz pierwszy przez zbiór $S_i$.\\
Stąd $\sum_{x\in S}c_x = \sum_{i=1}^{k}(n_{i-1}-n_i)\frac{1}{\vert S_i - (S_1 \cup ... \cup S_{i-1})\vert}$\\\\
Z wyboru zachłannego w linii 4 mamy $\vert S_i - (S_1 \cup ... \cup S_{i-1})\vert \leq \vert S - (S_1 \cup ... \cup S_{i-1})\vert = n_{i-1}$ dla dowolnego $S \in \mathcal{F} - \lbrace S_1,\ ...,\ S_{i-1}\rbrace$\\
\tab zatem\\
$\sum_{x\in S} c_x = \sum_{i=1}^{k} (n_{i-1}-n_i)\frac{1}{\vert S_i - (S_1 \cup ... \cup S_{i-1})\vert} \leq \sum_{i=1}^{k}(n_{i-1}- n_i)\frac{1}{n_{i-1}}=\sum_{i-1}^{k}\sum_{j=n_{i+1}}^{n_{i-1}}\frac{1}{n_{i-1}}\leq^{j\leq n_{i-1}}$\\
$\leq \sum_{i-1}^{k}\sum_{j=n_{i+1}}^{n_{i-1}}\frac{1}{j}=\sum_{i=1}^{k}(\sum_{j=1}^{n_{i-1}} \frac{1}{j}- \sum_{j=1}^{n_i} \frac{1}{j}) = \sum_{i=1}^{k} (H(n_{i-1}) - H(n_i)) = $\\
$ = H(n_0) - H(n_1) + H(n_1) - ... - H(n_k) = H(n_0) - H(n_k) = H(\vert S\vert ) - H(0) = H(\vert S\vert )$\\\\
Koszt przydzielony pokryciu $\mathcal{A}^{\ast}$ wynosi \\
\tab $\sum_{S\in \mathcal{A}^{\ast}}\sum_{x\in S}c_x$\\\\
Skoro $\mathcal{A}^{\ast}$ jest pokryciem X to każdy $x \in X$ wystąpi co najmniej raz w tej sumie,\\
więc\\
\tab $\sum_{S\in \mathcal{A}^{\ast}}\sum_{x\in S}c_x \geq \sum_{x\in X}c_x = \vert \mathcal{A}\vert$\\
więc\\
\tab $\vert \mathcal{A}\vert \leq \sum{S\in \mathcal{A}^{\ast}}\sum_{x\in S}c_x \leq \sum_{S\in \mathcal{A}^{\ast}}H(\vert S\vert )\leq \vert \mathcal{A}^{\ast}\vert max\lbrace H(\vert S\vert):\ S\in \vert \mathcal{A}^{\ast}\vert\rbrace\leq$\\
\tab $\leq\vert\mathcal{A}^{ast}\vert H(max\lbrace\vert S\vert :\ S\in \mathcal{F}\rbrace$\\\\
Wniosek: Algorytm GSC jest algorytmem $(ln\vert X \vert + 1)$-aproksymacyjnym\\
Dowód: $H(n)\leq lnn+1$\\
\tab $H(max\lbrace\vert S\vert :S\in \mathcal{F})\leq H(\vert X\vert )\leq ln\vert X\vert + 1$