\subsection{Znajdowanie par najbliższych punktów}
$Q$ - zbiór $n\geq 2$ punktów na płaszczyźnie 

$$p_1= (x_1,y_1)\ p_2= (x_2,y_2)$$
$$d(p_1,p_2)= \sqrt{(x_1-x_2)^2+(y_1-y_2)^2}$$

Należy znaleźć parę punktów z Q w minimalnej odległości euklidesowej.

Algorytm siłowy (Brute Force) - sprawdza wszystkie pary punktów, działa w czasie $\Theta(\binom{n}{2})=\Theta(n^2)$. Pokażemy algorytm typu "dziel i zwyciężaj" działający w czasie $O(n \log n)$.

W każdym rekurencyjnym wywołaniu algorytmu, danymi wejściowymi jest zbiór $P\subset Q$ oraz tablice $X$ i $Y$, z których każda zawiera wszystkie punkty z $P$.

W $X$ punkty są posortowane niemalejąco względem pierwszej współrzędnej (x-owej), a w $Y$ niemalejąco względem drugiej współrzędnej (y-owej).

\subsubsection{Opis Algorytmu}
\begin{itemize}
\item Jeśli $\vert P \vert \leq 3$ to sprawdzamy wszystkie pary

\item Jeśli $\vert P \vert > 3$ to robimy następująco:

	
		\textbf{Dziel: } 		

		Dzielimy punkty z $P$ pionową prostą L na dwa podzbiory $P_L$ i $P_R$ na lewo i prawo od L tak, że $$\vert P_L \vert= \lceil \frac{\vert P \vert}{2}\rceil,\ \vert P_R \vert = \lfloor \frac{\vert P \vert}{2} \rfloor $$
	
		Tablicę X dzielimy na $X_L$ i $X_R$ zmieniając odpowiednio punkty z $P_L$ i $P_R$ posortowane niemalejąco względem współrzędnej x-owej. Analogicznie dzielimy Y na $Y_L$ i $Y_R$ zawierające odpowiednie punkty z $P_L$ i $P_R$ posortowane niemalejąco względem współrzędnej y-owej.

		\textbf{Zwyciężaj: }
		
Wywołujemy rekurencyjnie algorytm dla $P_L$ i $P_R$ i znajdujemy pary najmniej odległych punktów w $P_L$ i $P_R$. 

Niech $\delta_L$ i $\delta_R$, będą najmniejszymi odległościami znalezionymi dla $P_L$ i $P_R$. 

Niech $\delta=min(\delta_L,\delta_R)$
		
		\textbf{Połącz: } 

		Para najbliższych punktów w $P$ składa się z dwóch punktów należących do $P_L$ lub z dwóch punktów należących do $P_R$ lub z jednego punktu należącego do $P_L$, a drugiego należącego do $P_R$.
			
		Sprawdzimy czy istnieje para punktów taka, że jeden punkt należy do $P_L$, a drugi należy do $P_R$, a ich odległość jest mniejsza niż $\delta$.
		
		Jeśli taka para istnieje, to oba punkty tej pary leżą w pasie wokół prostej L o szerokości $2\delta$.
		
		W celu sprawdzenia czy taka para punktów istnieje wykonamy następujące operacje:
		\begin{enumerate}
			\item Tworzymy tablicę $Y'$ uzyskaną z $Y$ przez usunięcie punktów spoza paska o szerokości $2\delta$ wokół L. Tablica $Y'$ jest posortowana niemalejąco względem współrzędnej y-owej.
			\item Dla każdego punktu $p$ z $Y'$ sprawdzamy czy odległość $p$ od $p'$ jest mniejsza niż $\delta$ dla 7 następnych punktów $p'$ w tablicy $Y'$. Algorytm oblicza najmniejszą z tych odległości $\delta'$. Zapamiętuje $\delta$ i parę punktów w tej odległości.
		\end{enumerate}
		
		Jeśli $\delta'<\delta$ to pionowy pas zawiera parę punktów w odległości mniejszej niż znaleziona w  wywołaniach rekurencyjnych i algorytm zawiera $\delta'$ i odpowiednią parę punktów.
		
		Jeśli $\delta'>\delta$ to zwracana jest $\delta$ z wywołań rekurencyjnych oraz odpowiednia para punktów.
\end{itemize}

\subsubsection{Poprawność}
Algorytm jest w sposób oczywisty poprawny za wyjątkiem kwestii dlaczego w fazie "połącz" w punkcie 2 wystarczy sprawdzić tylko 7 następujących po sobie punktów z tablicy $Y'$. Przypuśćmy, że $p$, $q$ są najmniej odległe w całym $P$, ich odległość wynosi $\delta'$.

Niech 
$p=(x_p,y_p)$ i $q=(x_q,y_q)$,\tab \tab $y_p \leq y_q$
pokażemy, że znajduje tę parę punktów.

\textbf{Dowód: }

Przypuśćmy, że algorytm nie wykryje tej pary. Wtedy w tablicy $Y'$ znajduje się pomiędzy $p$ i $q$ co najmniej 7 punktów.

Niech $r=(x_r,y_r)$ będzie jednym z nich.

Ponieważ $d(p,q)<\delta$, więc $y_q-y_p = \vert y_p - y_q \vert \leq \sqrt{(x_p-x_q)^2+(y_p-y_q)^2}=d(p,q)<\delta$

ponadto $y_p\leq y_c\leq y_q$, więc $y_r-y_p \leq y_q-y_p<\delta$, więc $y_r\leq y_p + \delta$

czyli co najmniej 9 punktów z $P$ znajduje się w prostokącie $=2\delta$x$\delta$ ograniczonym prostymi $y=y_p,\ y=y_p+\delta$

Wtedy w lewej lub w prawej połowie prostokąta jest co najmniej 5 punktów z $P$. Dla ustalenia uwagi uznajemy, że to lewy kwadrat, wtedy jednak w $P_L$ są dwa punkty odległe o mniej niż $\delta$, bo nie da się w kwadracie $\delta \times \delta$ rozmieścić 5 punktów wzajemnie odległych o co najmniej $\delta$.

Sprzeczność bo w $P_L$ najmniej odległe punkty znajdują się w odległości $\delta_L \geq \delta$.

\subsubsection{Pseudokod i Złożoność}
Na początku musimy posortować punkty z $Q$ w tablicach $X$ i $Y$, na co potrzeba $O(n \log n)$ czasu. Gdy mamy posortowane $X$ to podział na $X_L$ i $X_R$ zajmuje $O(n)$ czasu. Podobnie podział $Y$ na $Y_L$ i $Y_R$ oraz $P$ na $P_L$ i $P_R$. Można to łatwo zaimplementować tak, aby $Y_L$, $Y_R$ (oraz $X_L$, $X_R$) były nadal posortowane niemalejąco.

\begin{lstlisting}
length($Y_L$) $\leftarrow$ length($Y_R$) $\leftarrow$ 0
for i $\leftarrow$ 1 to length($Y$)
	if $Y(i) \in P_L$ 
    	length($Y_L$) $\leftarrow$ length($Y_L$)+1
    	$Y_L$(length($Y_L$)) $\leftarrow$ Y(i)
	else
    	length($Y_R$) $\leftarrow$ length($Y_R$)+1
    	$Y_R$(length($Y_R$)) $\leftarrow$ Y(i)
\end{lstlisting}

W fazie połącz tablicę $Y'$ otrzymujemy z $Y$ w czasie $O(n)$ podobnie jak poprzednio. $Y'$ będzie posortowane niemalejąco. 

$\delta = min(\delta_L,\ \delta_R)$ - minimalna odległość z wywołań rekurencyjnych. $x_R-x_L$ jest minimum prostej L.

\begin{lstlisting}
length($Y'$) $\leftarrow$ 0
for i $\leftarrow$ 1 to length($Y$)
	if $x_L-\delta \leq Y(i).x \leq x_L+\delta$ 
    	length($Y'$) $\leftarrow$ length($Y'$)+1
    	$Y'$(length($Y'$)) $\leftarrow$ Y(i)
\end{lstlisting}

Ponieważ dla każdego z $O(n)$ punktów z $Y'$ liczymy odległość do co najwyżej 7 punktów, więc najmniejszą odległość między punktami z $P_L$ i $P_R$ znajdziemy w czasie $O(n)$.

$T(n)$ - złożoność, algorytmu nie licząc sortowania $X$ i $Y$ na początku stąd otrzymujemy wzór:
$$T(n) = 2 \cdot T(\frac{n}{2})+O(n)$$
$a,b$ oraz $k$ mają wartości:
$$a=2 \tab b=2 \tab k=1 \tab 2=2^1$$
Zatem końcowa wartość złożoności to:
$$T(n) = O(n \log n)$$

Ponieważ sortowanie również zajmuje $O(n \log n)$ czasu, zatem cały algorytm działa w czasie $O(n \log n)$.
